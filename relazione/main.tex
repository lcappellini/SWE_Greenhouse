\documentclass{article}
\usepackage{graphicx} % Required for inserting images
\usepackage[margin=3cm]{geometry} % Adjust the margin as needed
\usepackage{tabularx}
\usepackage[table]{xcolor} % Pacchetto per colorare le celle
\usepackage[italian]{babel}
\usepackage{float}
\usepackage{hyperref}
\usepackage{caption}
\usepackage{subcaption}

% \newcommand{\code}[1]{\textbf{\textcolor{cyan}{#1}}}
\newcommand{\code}[1]{\texttt{#1}}

\title{GreenHouse \\ \vspace{0.5cm} {Software per la gestione di serre e piante}}
\author{Elion Karaboja \\ \and Lorenzo Cappellini}
\date{Maggio - Ottobre 2024} 

\begin{document}

\begin{titlepage}

\newcommand{\HRule}{\rule{\linewidth}{0.5mm}} % Defines a new command for the horizontal lines, change thickness here

%----------------------------------------------------------------------------------------
%	LOGO SECTION
%----------------------------------------------------------------------------------------
\center
\includegraphics[width=14cm]{resources/images/Icons/icon_unifilogo.png}\\[1cm] % Include a department/university logo - this will require the graphicx package
 
%----------------------------------------------------------------------------------------

\center % Center everything on the page

%----------------------------------------------------------------------------------------
%	HEADING SECTIONS
%----------------------------------------------------------------------------------------

\textsc{\LARGE Università degli Studi di Firenze}\\[0.5cm] % Major heading such as course name
\textsc{\Large Dipartimento di Ingegneria dell'Informazione}\\[0.5cm] % Minor heading such as course title

%----------------------------------------------------------------------------------------
%	TITLE SECTION
%----------------------------------------------------------------------------------------
\makeatletter
\HRule \\[1cm]
{ \huge \bfseries \@title}\\[0.7cm] % Title of your document
\HRule \\[1.5cm]
 
%----------------------------------------------------------------------------------------
%	AUTHOR SECTION
%----------------------------------------------------------------------------------------

\begin{minipage}{0.4\textwidth}
\begin{flushleft} \large
\emph{Autori:}\\
Elion Karaboja \newline Lorenzo Cappellini
\\[1.2em]
\emph{N° Matricola:}\\
7030984 \newline 7049027\\[1.2em]
\end{flushleft}
\end{minipage}
~
\begin{minipage}{0.4\textwidth}
\begin{flushright} \large
\emph{Corso principale:} \\
Ingegneria del Software  \\[1.2em]
\emph{Docente corso:} \\
Enrico Vicario
\end{flushright}
\end{minipage}\\[2cm]
\makeatother


\vfill % Fill the rest of the page with whitespace

\end{titlepage}

\tableofcontents

\newpage
\section{Introduzione}
Elaborato per il superamento dell’esame di Ingegneria del Software, appartenente al modulo Basi di Dati / Ingegneria del Software del corso di Laurea Triennale in Ingegneria Informatica dell’Università degli Studi di Firenze.
\\

\noindent Il progetto è stato sviluppato da Elion Karaboja e Lorenzo Cappellini (matricole 7030984 e 7049027) durante il periodo di Maggio - Settembre 2024 (a.a. 2023/2024).
\\

\noindent Il codice sorgente è disponibile su Github al seguente indirizzo:\\ 
\href{https://github.com/lcappellini/SWE_Greenhouse}{https://github.com/lcappellini/SWE\_Greenhouse}.

\subsection{Statement}
Il progetto modella un sistema di controllo e gestione di serre, e in particolare si occupa della cura e crescita delle piante in esse, al fine ultimo di venderle a clienti. 
I 3 spazi delle serre sono divisi in 4 settori in cui è presente un sistema di areazione, di illuminazione e controllo della temperatura, luce e umidità, e contengono 5 posizioni ciascuna nelle quali mettere le piante che saranno irrigate e monitorate. \\
La gestione delle piante è in gran parte automatizzata e possono essere monitorati sia i settori che le posizioni per ottenere un controllo \textit{real-time} da parte dell'Admin.\\ Ci sono operazioni che dovranno essere effettuate manualmente come la cura e il posizionamento delle piante nelle posizioni designate, che saranno eseguite da operatori. Tale sistema è adottato da un'azienda che permette ai clienti di acquistare e commissionare un certo numero di piante con consegna prevista entro un certo periodo.
\subsection{Archiettura e pratiche utilizzate}
Il software è stato sviluppato in Java, mentre per la gestione e il salvataggio dei dati è stato connesso un database PostgreSQL ed è stata utilizzata la libreria JDBC (Java DataBase Connectivity).
\\

\noindent Per mantenere una separazione delle responsabilità, la struttura del progetto è stata divisa in tre parti principali: Business Logic, Domain Model e ORM. Questi tre packages si occupano in modo distinto della logica di business, della rappresentazione dei dati e dell’accesso ai dati (Figura \ref{fig:diagram_packagedependency}):
\begin{itemize}
    \item \textbf{Business Logic}: contiene le classi che implementano la logica di business del sistema.
    \item \textbf{Domain Model}: contiene le classi che rappresentano le entità del sistema.
    \item \textbf{ORM}: contiene le classi che implementano l’Object-Relational Mapping. In questo modo è possibile rendere i dati persistenti e recuperarli dal database.
\end{itemize}
Per utilizzare il software è stata creata un’interfaccia da riga di comando (CLI) che permette di interagire con il sistema in modo semplice e intuitivo.
\\

\noindent Gli Use Case Diagram e i Class Diagram seguono lo standard UML (Unified Modeling Language) e sono stati realizzati con il software StarUML. Infine, per la parte di testing è stato utilizzato JUnit.
\\

\noindent Questa architettura prevede che l'utente esegua un'azione tramite l'interfaccia; questa richiama direttamente i metodi esposti nei controller della Business Logic; quest'ultima delega la creazione degli oggetti all'ORM che richiede i dati al database tramite JDBC e RDBMS; creati gli oggetti (definiti nel Domain Model), la Business Logic può usarli direttamente e portare a termine l'esecuzione della funzione desiderata. Vedi Figura \ref{fig:diagram_packagedependency}.

\begin{figure}[H]
    \centering
    \fbox{\includegraphics[scale=0.3]{resources/images/Diagrams/diagram_packagedependency.jpg}}
    \caption{Package Dependency Diagram}
    \label{fig:diagram_packagedependency}
\end{figure}

\section{Progettazione}

\subsection{Use Case Diagram}
Come descritto in precedenza, il sistema è stato progettato per permettere a clienti di ordinare e ritirare le piante desiderate, le quali sono gestite da operatori e monitorate da un Admin. Le Figure \ref{fig:diagram_usecases_client} e \ref{fig:diagram_usecases_worker} illustra il diagramma dei casi d'uso del sistema, includendo quelli relativi ai Clienti, agli Operatori e agli Admin.
\\



\begin{figure}[H]
    \centering
    \fbox{\includegraphics[scale=0.27]{resources/images/Diagrams/use_Case_cliente.jpg}}
    \caption{Diagramma degli Use Cases del Cliente}
    \label{fig:diagram_usecases_client}
\end{figure}
\begin{figure}[H]
    \centering
    \fbox{\includegraphics[scale=0.27]{resources/images/Diagrams/use_case_personale.jpg}}
    \caption{Diagramma degli Use Cases del Personale lavorativo}
    \label{fig:diagram_usecases_worker}
\end{figure}

\subsection{Use Case Template}
Qui di seguito sono elencati i template di alcuni casi d'uso implementati. Per ciascuno sono specificati dei dettagli: una breve descrizione, il livello del caso d'uso, gli attori coinvolti, le precondizioni, le post-condizioni, il flusso principale e i flussi alternativi.
\\

    \renewcommand{\arraystretch}{1.5}
    
    \begin{table}[h]
        \begin{tabularx}{\textwidth}{ | l  X | }
            \rowcolor{lightgray!70}
            \hline
            \textbf{Use Case \#1} & \textbf{Accedi al Sistema (Sign-in)}\\[0.5ex]
            \textbf{Brief Description} & L'utente accede al sistema tramite le proprie credenziali \\
            \rowcolor{blue!10}
            \textbf{Level} & User Goal \\
            \textbf{Actors} & Admin, Cliente, Operatore \\
            \rowcolor{blue!10}
            \textbf{Pre-Conditions} & L'utente deve essere nella pagina dedicata al proprio accesso (\hyperref[fig:mockup_1]{Mockup \#1}) \\
            \textbf{Basic Flow} & 1) L'utente inserisce le proprie credenziali\\
            & 2) L'utente invia le proprie credenziali\\
            & 3) Il sistema verifica le credenziali\\
            & 4) Il sistema autentica l'utente (Test \hyperref[test_3]{\#3}, \hyperref[test_6]{\#6} e \hyperref[test_9]{\#9})  \\        
            \rowcolor{blue!10}
            \textbf{Alternative Flow} & 3a) Se le credenziali sono errate o l'utente non è registrato, il sistema restituisce un messaggio di errore e permette di ritentare (Test \hyperref[test_4]{\#4}, \hyperref[test_5]{\#5}, \hyperref[test_7]{\#7}, \hyperref[test_8]{\#8}, \hyperref[test_10]{\#10}, \hyperref[test_11]{\#11})\\
            \textbf{Post-Conditions} & L'utente è autenticato nel sistema e ha accesso alle sue funzioni \\
            \hline
        \end{tabularx}
        \label{use_case_1}
        \caption{Use Case \#1 (Accedi al Sistema)}
    \end{table}
    
    
    \begin{table}[p]
        \begin{tabularx}{\textwidth}{ | l  X | }
            \hline
            \rowcolor{lightgray!70}
            \textbf{Use Case \#2} & \textbf{Registrati nel Sistema (Sign-up)}\\ [0.5ex]
            \textbf{Brief Description} & Il cliente si registra nel sistema creando un nuovo account \\
            \rowcolor{blue!10}
            \textbf{Level} & User Goal \\
            \textbf{Actors} & Cliente \\
            \rowcolor{blue!10}
            \textbf{Pre-Conditions} & Il cliente deve essere nella pagina di registrazione del cliente (Mockup \hyperref[fig:mockup_2]{\#2})\\
            \textbf{Basic Flow} & 1) Il cliente fornisce le informazioni richieste\\
            & 2) Il cliente conferma la registrazione\\
            & 3) Il sistema verifica i dati\\
            & 4) Il sistema crea un nuovo account per il cliente\\  
            \rowcolor{blue!10}
            \textbf{Alternative Flow} & 3a) Se il cliente fornisce dati non validi o già presenti (di un altro account), il sistema mostra un messaggio di errore\\
            \textbf{Post-Conditions} & Il cliente deve comunque effettuare il login per accedere al sistema\\
            \hline
        \end{tabularx}
        \label{use_case_2}
        \caption{Use Case \#2 (Registrati  al Sistema)}
    \end{table}
    
    
    \begin{table}[p]
        \begin{tabularx}{\textwidth}{ | l  X | }
            \rowcolor{lightgray!70}
            \hline
            \textbf{Use Case \#3} & \textbf{Richiedi nuovo Ordine} \\[0.5ex]
            \textbf{Brief Description} & Il cliente crea un nuovo Ordine \\
            \rowcolor{blue!10}
            \textbf{Level} & User Goal \\
            \textbf{Actors} & Cliente \\
            \rowcolor{blue!10}
            \textbf{Pre-Conditions} & Il cliente deve aver fatto l'accesso e essere nella pagina "Ordini" \\
            \textbf{Basic Flow} & 1) Il cliente seleziona l'opzione per creare un nuovo ordine\\
            & 2) Il cliente sceglie il tipo di pianta e la quantità desiderata (Mockup \hyperref[fig:mockup_3]{\#3})\\
            & 3) Il cliente sceglie se aggiungere altre piante o se concludere l'ordine\\
            & 4) Il sistema registra il nuovo ordine (Test \hyperref[test_12]{\#12})\\
            \rowcolor{blue!10}
            \textbf{Alternative Flow} & 2a) Se il cliente inserisce valori non validi, il sistema mostra un messaggio di errore e richiede l'input  \\
            \rowcolor{blue!10}
            & 4a) Se l'ordine non può essere preso in carico, il sistema restituisce un messaggio (Test \hyperref[test_13]{\#13})\\
            \textbf{Post-Conditions} & Il nuovo ordine è stato registrato nel sistema \\
            \hline
        \end{tabularx}
        \label{use_case_3}
        \caption{Use Case \#3 (Richiedi nuovo Ordine)}
    \end{table}
    
    
    \begin{table}[p]
        \begin{tabularx}{\textwidth}{ | l  X | }
            \rowcolor{lightgray!70}
            \hline
            \textbf{Use Case \#4} & \textbf{Paga e Ritira Ordine }\\[0.5ex]
            \textbf{Brief Description} & Il cliente paga e ritira il proprio Ordine\\
            \rowcolor{blue!10}
            \textbf{Level} & User Goal \\
            \textbf{Actors} & Cliente \\
            \rowcolor{blue!10}
            \textbf{Pre-Conditions} & Il cliente è autenticato ed è nella sua pagina "Ordini" \\
            \textbf{Basic Flow} & 1) Il cliente seleziona l'opzione per pagare e ritirare l'ordine\\
            & 2) Il sistema mostra una lista di ordini pronti per essere ritirati\\
            & 3) Il cliente inserisce l'id dell'ordine scelto\\
            & 4) Il sistema conferma il pagamento e il ritiro di tale ordine (Test \hyperref[test_14]{\#14})\\
            \rowcolor{blue!10}
            \textbf{Alternative Flow} & 3a) Se l'utente fornisce un id sconosciuto o l'id di un ordine non pronto, il sistema mostra un messaggio di errore\\
            \textbf{Post-Conditions} & Il sistema libera le posizioni associate e rimuove le piante ritirate \\
            \hline
        \end{tabularx}
        \label{use_case_4}
        \caption{Use Case \#4 (Paga e Ritira Ordine)}
    \end{table}
    
    
    \begin{table}[p]
        \begin{tabularx}{\textwidth}{ | l  X | }
            \rowcolor{lightgray!70}
            \hline
            \textbf{Use Case \#5} & \textbf{Pianta Ordine }\\[0.5ex]
            \textbf{Brief Description} & L'Operatore semina le piante associate a un Ordine \\
            \rowcolor{blue!10}
            \textbf{Level} & User Goal\\
            \textbf{Actors} & Operatore \\
            \rowcolor{blue!10}
            \textbf{Pre-Conditions} & L'Operatore deve essere sulla sua dashboard\\
            \textbf{Basic Flow} & 1) L'Operatore sceglie l'opzione "Pianta Ordine"\\
            & 2) Il sistema mostra una lista di Ordini da piantare (Mockup      \hyperref[fig:mockup_4]{\#4})\\
            & 3) L'Operatore sceglie l'id dell'Ordine che vuole piantare\\
            & 4) Il sistema segnala all'operatore di posizionare le piante dell'ordine in posizioni disponibili (Test \hyperref[test_20]{\#20})\\
            \rowcolor{blue!10}
            \textbf{Alternative Flow} & 3a) Se il sistema non trova l'ordine con tale id o l'ordine è già stato posizionato, mostra un messaggio di errore\\
            \textbf{Post-Conditions} & Il sistema genera i Posizionamenti dell'ordine e modifica lo stato dell'Ordine\\
            \hline
        \end{tabularx}
        \label{use_case_5}
        \caption{Use Case \#5 (Pianta Ordine)}
    \end{table}
    
    
    \begin{table}[p]
        \begin{tabularx}{\textwidth}{ | l  X | }
            \rowcolor{lightgray!70}
            \hline
            \textbf{Use Case \#6} & \textbf{Completa Ordine} \\[0.5ex]
            \textbf{Brief Description} & Completa un Ordine (le piante sono cresciute e pronte alla consegna)\\
            \rowcolor{blue!10}
            \textbf{Level} & User Goal \\
            \textbf{Actors} & Operatore \\
            \rowcolor{blue!10}
            \textbf{Pre-Conditions} & L'Operatore deve essere sulla sua dashboard\\
            \textbf{Basic Flow} & 1) L'Operatore sceglie l'opzione "Completa Ordine"\\
            & 2) Il sistema mostra una lista di Ordini da completare\\
            & 3) L'Operatore inserisce l'id dell'Ordine da completare\\
            & 4) Il sistema imposta l'Ordine come completato e quindi pronto al ritiro (Test \hyperref[test_15]{\#15})\\
            \rowcolor{blue!10}
           \textbf{Alternative Flow} & 3a) Se il sistema non trova l'ordine con tale id o l'ordine è già stato completato, mostra un messaggio di errore (Test \hyperref[test_16]{\#16})\\
            \textbf{Post-Conditions} & Il cliente adesso può pagare e ritirare l'ordine\\
            \hline
        \end{tabularx}
        \label{use_case_6}
        \caption{Use Case \#6 (Completa Ordine)}
    \end{table}
    
    
    
    \begin{table}[p]
        \begin{tabularx}{\textwidth}{ | l  X | }
            \rowcolor{lightgray!70}
            \hline
            \textbf{Use Case \#7} & \textbf{Check-Up Piante }\\[0.5ex]
            \textbf{Brief Description} & L'Operatore verifica lo stato delle piante\\
            \rowcolor{blue!10}
            \textbf{Level} & User Goal\\
            \textbf{Actors} & Operatore \\
            \rowcolor{blue!10}
            \textbf{Pre-Conditions} & L'Operatore deve essere sulla dashboard\\
            \textbf{Basic Flow} & 1) L'Operatore sceglie l'opzione "Check-Up Piante"\\
            & 2) Per ogni Pianta del sistema viene eseguito un controllo che valuta se la pianta ha bisogno di cure (Test \hyperref[test_17]{\#17})\\
            & 3) Se sono presenti piante che hanno bisogno, queste vengono curate (Test \hyperref[test_19]{\#19})\\
            & 4) Il Check-Up è terminato\\
            & 5) Il sistema salva l'operazione effettuata dall'operatore nel database \\
            \rowcolor{blue!10}
            \textbf{Alternative Flow} & 2a) Se non sono presenti piante su cui è possibile fare il check-up, il sistema mostra un errore (Test \hyperref[test_18]{\#18})\\
            \textbf{Post-Conditions} & Le Piante sono in salute e possono continuare a crescere\\
            \hline
        \end{tabularx}
        \label{use_case_7}
        \caption{Use Case \#7 (Check-Up Piante)}
    \end{table}
    
    
    \begin{table}[h]
        \begin{tabularx}{\textwidth}{ | l  X | }
            \rowcolor{lightgray!70}
            \hline
            \textbf{Use Case \#8} & \textbf{Monitora Settore }\\[0.5ex]
            \textbf{Brief Description} & L'Admin visualizza i parametri dei Sensori e degli Attuatori in tempo reale \\
            \rowcolor{blue!10}
            \textbf{Level} & User Goal \\
            \textbf{Actors} & Admin \\
            \rowcolor{blue!10}
            \textbf{Pre-Conditions} & L'Admin deve essere autenticato e sulla dashboard dell'admin\\
            \textbf{Basic Flow} & 1) L'Admin sceglie l'opzione "Monitora Settore"\\
            & 2) Il sistema mostra una lista di settori monitorabili\\
            & 3) L'Admin inserisce l'id del Settore desiderato\\
            & 4) Il sistema restituisce i valori dei sensori in tempo reale e lo stato ON/OFF dei rispettivi attuatori\\
            & 5) L'Admin interrompe il monitoraggio\\
            \rowcolor{blue!10}
            \textbf{Alternative Flow} & 4a) Il sistema non trova il Settore con l'id richiesto e mostra un messaggio di errore\\
            \textbf{Post-Conditions} & L'Admin ritorna alla dashboard \\
            \hline
        \end{tabularx}
        \label{use_case_8}
        \caption{Use Case \#8 (Monitora Settore)}
    \end{table}


\subsection{Mock-ups}
Ecco di seguito alcuni possibili Mock-ups relativi alle interfacce grafiche del sistema, realizzati mediante il sito web \href{https://www.figma.com}{figma.com} . \
Di seguito è una breve spiegazione per ogni Mock-up:
\begin{itemize}
    \item Mockup \#1 (Prototipo della pagina di Sign in): Permette di inserire Email e Password e di scegliere se l'utente è un cliente o fa parte del personale, per effettuare il login;
    \item Mockup \#2 (Prototipo della pagina di Sign up): Permette di inserire i dati necessari alla registrazione di un cliente nel sistema;
    \item Mockup \#3 (Prototipo della pagina per richiedere nuovi ordini): Mostra al cliente le piante disponibili per essere scelte (tramite i pulsanti "+" e "-") e aggiunte ad un ordine;
    \item Mockup \#4 (Prototipo della pagina per piantare gli ordini): Mostra all'operatore gli ordini da piantare, in particolare i dati dell'ordine mostrati sono la data di consegna prevista, le piante e l'id del cliente.
\end{itemize}

\begin{figure}[H]
    \centering
    \includegraphics[scale=0.37]{resources/images/Mockups/mockup_1.png}
    \caption{Prototipo della pagina di Sign in - Mockup \#1}
    \label{fig:mockup_1}
\end{figure}
\begin{figure}[H]
    \centering
    \includegraphics[scale=0.35]{resources/images/Mockups/mockup_2.png}
    \caption{Prototipo della pagina di Sign up - Mockup \#2}
    \label{fig:mockup_2}
\end{figure}
\begin{figure}[H]
    \centering
    \includegraphics[scale=0.35]{resources/images/Mockups/mockup_3.png}
    \caption{Prototipo della pagina per richiedere nuovi ordini - Mockup \#3}
    \label{fig:mockup_3}
\end{figure}
\begin{figure}[H]
    \centering
    \includegraphics[scale=0.35]{resources/images/Mockups/mockup_4.png}
    \caption{Prototipo della pagina per piantare gli ordini - Mockup \#4}
    \label{fig:mockup_4}
\end{figure}

\subsection{Navigation Diagram}
\label{subsec:nav-diagram}
Il seguente diagramma (Figura \ref{fig:diagram_navigation}) rappresenta quelle che sono le pagine principali del sistema e le possibili azioni che l’utente può compiere nell'interfaccia CLI proposta. Sono anche rappresentati i modi con cui si può navigare tra le varie pagine.
Alcune delle pagine sono:
\begin{itemize}
    \item \textbf{Greenhouse}: l'utente sceglie la propria area di appartenenza.
    \item \textbf{Cliente Dashboard}: il cliente può scegliere se entrare nella pagina degli Ordini e quindi crearne uno nuovo, pagarne uno pronto o controllarli, oppure nella pagina Profilo dove può visualizzare o modificare i dati del proprio profilo.
    \item \textbf{Operatore Dashboard}: L'operatore può eseguire i suoi compiti tra quelli elencati quindi piantare un ordine, completarlo per poter essere ritirato dal cliente oppure può eseguire un controllo sullo stato delle piante.
    \item \textbf{Admin Dashboard}: L'admin ha come funzione principale quella di monitorare i settori e le relative posizioni occupate grazie ai sensori e attuatori. Inoltre può visualizzare, come l'operatore, le tabelle di ordini, piante e clienti.
\end{itemize}
\begin{figure}[H]
    \centering
    \fbox{\includegraphics[scale=0.4]{resources/images/Diagrams/diagram_navigation.jpg}}
    \caption{Navigation Diagram}
    \label{fig:diagram_navigation}
\end{figure}

\subsection{Class Diagram}
Vista la divisione strutturale in 3 packages, sono stati realizzati 3 diagrammi delle classi distinti, uno per ogni package:

\begin{itemize}
    \item \textbf{Business Logic} (Figura \ref{fig:diagram_businesslogic}): contiene le classi che implementano la logica di business del sistema. Ad ogni utente è stato associato un "Controller", che espone i metodi che esso può utilizzare. Qundi sono presenti i seguenti controller: quello che gestisce l’accesso e la registrazione dei nuovi Clienti (LoginClienteController), degli Admin e degli Operatori (LoginPersonaleController), quello che permette ai Clienti di richiedere nuovi ordini, visualizzarli e di ritirarli (ClienteController), quello che gestisce il monitoraggio e visualizzazione dell'impianto da parte dell'Admin (AdminController), infine il controller che segnala le operazioni dell'operatore come piantare un ordine, completarlo o fare il checkup delle piante (OperatoreController). \
    È presente anche \code{AdminExtraController} che contiene metodi ausiliari al reset del database (utilizzati soprattutto nei test).\\
    Una descrizione più dettagliata di ogni controller è presente nella sezione \ref{subsec:business-logic}.
        
    
    \item \textbf{Domain Model} (Figura \ref{fig:diagram_domainmodel}): contiene le classi che rappresentano le entità del sistema ovvero: Ordine, Utente, Cliente, Operatore, Admin, Pianta e il Posizionamento (quest'ultimo funge da Mapper tra Pianta, Ordine e Posizione). In più contiene anche un altro package chiamato Impianto nel quale sono presenti le classi delle entità che compongono la struttura dell'azienda quali: Spazio, Settore e Settore che ne rappresentano l'entità immobile, Attuatore e Sensori che ne rappresentano le entità operative (nello specifico Climatizzatore, Lampada, Irrigatore sono attuatori; IgrometroAria, IgrometroTerra, Termometro e Fotosensore sono sensori).
        
    
    \item \textbf{ORM} (Figura \ref{fig:diagram_orm}): contiene le classi che implementano l’Object-Relational Mapping, quindi contiene una classe per ogni entità del sistema: \code{ClienteDAO}, \code{OperatoreDAO}, \code{AdminDAO}, \code{OrdineDAO}, \code{PiantaDAO}, \code{PosizionamentoDAO}, \code{SettoreDAO}, \code{PosizioneDAO}, \code{AttuatoreDAO} e \code{SensoreDAO}. In più contiene anche la classe \code{ConnectionManager} che si occupa di gestire la connessione al database.\
    È importante notare che \code{OrdineDAO} usa al suo interno (nel metodo \code{get}) \code{PiantaDAO} in quanto è stata affidata a lui la responsabilità di inizializzare anche la lista di piante (attributo della classe \code{Ordine}) e quindi le piante in essa contenute.
        
\end{itemize}
\begin{figure}[H]
    \centering
    \fbox{\includegraphics[scale=0.4]{resources/images/Diagrams/class_diagram_businesslogic.jpg}}
    \caption{Class Diagram - BusinessLogic}
    \label{fig:diagram_businesslogic}
\end{figure}
\begin{figure}[H]
    \hspace{-1cm}
    \fbox{\includegraphics[scale=0.5]{resources/images/Diagrams/class_diagram_domainmodel.jpg}}
    \caption{Class Diagram - DomainModel}
    \label{fig:diagram_domainmodel}
\end{figure}
\begin{figure}[H]
    \centering
    %\hspace{-1.7cm}
    \fbox{\includegraphics[scale=0.37]{resources/images/Diagrams/class_diagram_orm.jpg}}
    \caption{Class Diagram - ORM}
    \label{fig:diagram_orm}
\end{figure}



\subsection{ER Diagram e modello relazionale}
Il database è stato progettato seguendo il modello relazionale e facendo attenzione alle relazioni che legano le entità (Figura \ref{fig:diagram_er}).\\
Sono state quindi definite le seguenti tabelle:
\begin{itemize}
    \item \textbf{Cliente}: rappresenta l'entità Cliente;
    \item \textbf{Ordine}: rappresenta l’entità Ordine e risolve la relazione "richiedi" con Cliente (un Cliente può fare più Ordini, quindi gli Ordini contengono l'id del Cliente che li ha effettuati);
    \item \textbf{Pianta}: rappresenta l’entità Pianta e risolve la relazione "ha" con Ordine, in quanto un Ordine possiede un insieme di Piante, quindi ogni pianta possiede l'id dell'Ordine a cui appartiene;
    \item \textbf{Admin}: rappresenta l’entità Admin;
    \item \textbf{Operatore}: rappresenta l’entità Operatore;
    \item \textbf{Spazio}: rappresenta l’entità Spazio;
    \item \textbf{Settore}: rappresenta l’entità Settore e contiene l'id dello Spazio a cui appartiene, e gli id dei Sensori e degli Attuatori situati in esso (relazioni di appartenenza);
    \item \textbf{Posizione} rappresenta l’entità Posizione, prende parte alla relazione "Posizionamento" e contiene gli id dei sensori e attuatori che gli appartengono;
    \item \textbf{Posizionamento}: rappresenta l’entità Posizionamento che risolve la relazione di mapping tra Ordine, Pianta e Posizione;
    \item \textbf{Attuatore}: rappresenta l’entità Attuatore.
    \item \textbf{Sensore}: rappresenta l’entità Sensore.
    \item \textbf{Operazione}: rappresenta l’entità Operazione;
\end{itemize}
\begin{figure}[H]
    \centering
    \fbox{\includegraphics[scale=0.3]{resources/images/Diagrams/diagram_er.jpg}}
    \caption{ER Diagram}
    \label{fig:diagram_er}
\end{figure}

\begin{figure}[H]
    \centering
    \fbox{\includegraphics[scale=0.3]{resources/images/Diagrams/diagram_modellorelazionale.png}}
    \caption{Modello Relazionale}
    \label{fig:diagram_modellorelazionale}
\end{figure}

\section{Implementazione}
Il codice è suddiviso in 3 "packages", che suddividono le classi in base alle loro funzionalità:

\subsection{Domain Model}
Contiene tutte le classi che rappresentano le entità del sistema e le relative funzioni associate.

\subsubsection{Utente}
La classe astratta Utente rappresenta un utente del sistema. 
Qui di seguito sono elencate le 3 classi che ereditano Utente e rappresentano gli agenti reali che accedono e interagiscono con il sistema stesso.

\begin{itemize}
    \item \textbf{Cliente:}
    Rappresenta un Cliente dell'azienda, interessato a effettuare un acquisto di piante su ordinazione. I campi della classe sono \code{id}, \code{nome}, \code{cognome}, \code{email} e \code{password}.

    \item \textbf{Admin:} Identifica l'Admin dell'azienda e, come \code{Cliente}, contiene \code{id}, \code{nome}, \code{cognome}, \code{email} e \code{password}. 
    La sua interfaccia gli permette di visualizzare ordini, piante e clienti e monitorare i settori dell'impianto.

    \item \textbf{Operatore:}
    Rappresenta un \code{Operatore} che lavora nell'azienda. È un utente, che quindi può accedere con le proprie credenziali, come un Utente qualsiasi, però può essere visto anche come un Attuatore con più funzionalità. Infatti può piantare un ordine, prepararlo quando è pronto oppure può controllare lo stato delle piante. I suoi attributi sono \code{id, nome, cognome, email, password e working}.
\end{itemize}

\subsubsection{Pianta}
    Rappresenta una \code{pianta} presente nella serra, viene definita con \code{tipo di Pianta}, una \code{descrizione} di essa, la \code{dataInizio} in cui è stata piantata, lo \code{stato} attuale, il suo \code{costo} e il numero di \code{giorni} necessari alla sua crescita.\\
    Il metodo \code{cura} effettua una cura (simulata) da parte di un Operatore nel caso in cui la Pianta ne necessiti.
    
\subsubsection{Ordine}
    Identifica un \code{ordine} di certe \code{piante} effettuato da un \code{cliente} e i relativi dettagli come la \code{data di consegna}, il prezzo \code{totale}, un \code{id} univoco e lo \code{stato} attuale.
    I suoi metodi sono tutti getters e setters.\\
    Lo \code{stato} dell'Ordine può essere:
    \begin{itemize}
        \item "da piantare": l'ordine è stato accettato dal sistema ed è stato preso in carico;
        \item "posizionato": le piante richieste nell'ordine sono state seminate nelle posizioni;
        \item "da completare": le piante sono cresciute e sono pronte per essere tolte dalle posizioni;
        \item "da ritirare": le piante possono essere ritirate dal cliente;
        \item "ritirato": l'ordine è stato pagato e ritirato dal cliente.
    \end{itemize}
    I passaggi di stato sono attuati attraverso precise funzioni del sistema (vedi Figura \ref{fig:diagram_statiordine}). In particolare \code{modificaStatoOrdine} permette di impostare uno stato arbitrario ed è utilizzata esclusivamente per "debug", però è indispensabile dato che la parte di simulazione della crescita della pianta e quindi del passaggio dell'ordine a stato "da completare" non è stata implementata.

    \begin{figure}[H]
        \centering
        \fbox{\includegraphics[scale=0.2]{resources/images/Diagrams/stati_ordini.jpg}}
        \caption{Diagramma degli Stati di un Ordine}
        \label{fig:diagram_statiordine}
    \end{figure}


\subsubsection{Impianto}
\begin{itemize}
    \item \textbf{Spazio:}
    Rappresenta uno \code{Spazio} presente nella GreenHouse.
    È identificato da un \code{id} e contiene un insieme di Settori.
    \item \textbf{Settore:}
    Rappresenta una zona chiusa e isolata (fisicamente) avente al suo interno un \code{Termometro}, \code{IgrometroAria} e un \code{Fotosensore} che rilevano il "clima" all'interno e degli attuatori come il \code{Climatizzatore} e la \code{Lampada}. Contiene anche una lista di \code{Posizioni} e un \code{id}.
    \item \textbf{Posizione:}
    Identifica una \code{Posizione} in cui è stata piantata una singola \code{Pianta}.
    Contiene un proprio \code{Irrigatore} e un \code{IgrometroTerra}, un \code{id} univoco e due "flag" boolean \code{assegnata} e \code{occupata} che indicano lo stato della Posizione.\\
    Una Posizione è "assegnata" quando un Ordine è preso in carico e ci verrà piantata una Pianta da un Operatore.\\
    La flag "occupata" indica appunto che la semina è avvenuta e tale Posizione è occupata fisicamente.
    \item \textbf{Posizionamento:}
    Mapper con \code{id} che associa una \code{Pianta} con una \code{Posizione} e un \code{Ordine}.
\end{itemize}

\subsubsection{Sensori}
Classe astratta che definisce le proprietà e funzioni che deve avere un Sensore di qualunque tipo, ovvero un dispositivo capace di misurare certe grandezze fisiche di un Settore o di una Posizione (Pianta), con appositi metodi. Ha un \code{id} e misura un certo \code{valore}.\\
Le classi che ereditano questa classe astratta dovranno implementare il metodo \code{misura}.

\begin{itemize}
    \item \textbf{Termometro:}
    Sensore che misura la temperatura (dell'aria) del Settore.
    \item \textbf{Fotosensore:}
    Sensore che misura l'intensità luminosa della luce presente nel Settore in cui è posto.
    \item \textbf{IgrometroAria:}
    Sensore che misura l'umidità presente nell'aria. È necessario e sufficiente soltanto uno di questi sensori per ogni Settore, in quanto le Posizioni (Piante) in esso condividono la stessa aria.    
    \item \textbf{IgrometroTerra:}
    Sensore che misura l'umidità del terreno in cui è piantata una certa pianta. Ogni Posizione ha il suo.
\end{itemize}

\subsubsection{Attuatori}
    Classe astratta che identifica un dispositivo in grado di effettuare azioni al fine di modificare parametri fisici del Settore o Posizione in cui esso è situato.
    Può essere attivato e disattivato con appositi metodi.\\
    Come per Sensore, è creata come classe astratta per impostare un' "interfaccia" comune di riferimento a tutte le sottoclassi e permette una gestione più semplice dei diversi attuatori.\\
    Le sottoclassi dovranno implementare il metodo astratto \code{esegui}.
    \begin{itemize}
        \item \textbf{Climatizzatore:}
        Attuatore che regola la temperatura e l'umidità dell'aria. È sufficiente un Climatizzatore per ogni settore.
        \item \textbf{Irrigatore:}
        Attuatore che irriga la Pianta presente nella Posizione in cui è situato tale dispositivo.
        \item \textbf{Lampada:}
        Attuatore che regola un livello di luminosità della luce di un Settore, per soddisfare il fabbisogno delle Piante.
    \end{itemize}

\newpage

\subsection{Business Logic}
\label{subsec:business-logic}
Contiene tutte le classi e le relative funzionalità che hanno lo scopo di gestione delle varie entità del sistema.

\subsubsection{LoginClienteController}
Questa classe è utilizzata per effettuare il login del Cliente nel sistema. Ha come metodi \code{Accedi}, che verifica se le credenziali fornite sono valide, e \code{Registrati} che registra le credenziali di un nuovo cliente nel sistema.

\subsubsection{LoginPersonaleController}
Come \code{LoginClienteController}, si occupa del login del personale lavorativo della Greenhouse. Fa distinzione internamente tra i metodi \code{loginAdmin} e \code{loginOperatore}, anche se sono implementati in maniera molto simile.

\subsubsection{ClienteController}
La classe \code{ClienteController} definisce un'interfaccia per il cliente, che quindi, secondo gli Use Cases previsti, può: 
\begin{itemize}
    \item aggiornare i dati del proprio profilo, utilizzando \code{aggiornaProfilo};
    \item richiedere un nuovo ordine, con \code{richiediNuovoOrdine};
    \item pagare e ritirare un ordine, con \code{pagaEritiraOrdine}
\end{itemize}
È presente anche un altro metodo, \code{getOrdini}, che ottiene una lista di ordini associati a tale cliente ed eventualmente secondo altri criteri: questo serve quando è necessario mostrare gli ordini nell'interfaccia del programma.

\subsubsection{OperatoreController}
Questa classe presenta metodi che sono chiamati in maniera "diretta" dall'operatore per svolgere alcuni compiti che gli competono, come:
\begin{itemize}
    \item \code{piantaOrdine}: pianta un ordine selezionato se nello stato "da piantare";
    \item \code{completaOrdine}: completa un ordine e imposta il suo stato come "da ritirare";
    \item \hyperref[fig:snippet_checkuppiante_curapianta]{\code{checkupPiante}}: simula il lavoro dell'operatore nel controllare manualmente le piante e ritorna una lista delle piante che necessitano di cure;
    \item \hyperref[fig:snippet_checkuppiante_curapianta]{\code{curaPianta}}: simula la cura di una pianta da parte dell'operatore;
    \item \code{generaStatoPiante}: stabilisce randomicamente se una pianta ha bisogno di cure (simulando, come se dipendesse da agenti esterni dell'ambiente imprevedibili)
    \item getters di Clienti, Ordini e Piante utili per mostrare i relativi dati sull'interfaccia utente.
\end{itemize}

\begin{figure}[H]
    \centering
    \includegraphics[scale=0.5]{resources/images/Snippets/snippet_checkuppiante_curapianta.png}
    \captionsetup{labelformat=empty, labelsep=none}
    \caption{Snippet 1: metodi \code{checkupPiante()} e \code{curaPianta()} nella classe \code{OperatoreController}}
    \label{fig:snippet_checkuppiante_curapianta}
\end{figure}

\subsubsection{AdminController}
In questa classe è presente il metodo \hyperref[fig:snippet_monitorasettore]{\code{monitoraSettore}} chiamato quando viene avviato un monitoraggio da parte dell'Admin. Se richiesto è lanciato anche \code{monitoraPosizioniBySettoreId} che fornisce dati su sensori e attuatori relativi alle Posizioni del Settore desiderato.\\
Anche qui sono presenti getters utili alla visualizzazione dei dati.

\begin{figure}[H]
    \centering
    \includegraphics[scale=0.5]{resources/images/Snippets/snippet_monitorasettore.png}
    \captionsetup{labelformat=empty, labelsep=none}
    \caption{Snippet 2: metodo \code{monitoraSettore()} nella classe \code{AdminController}}
    \label{fig:snippet_monitorasettore}
\end{figure}

\subsubsection{AdminExtraController}
Infine con i metodi di AdminExtraController l’admin può fare il reset del database con \hyperref[fig:snippet_resetdatabase]{resetDatabase()} e inserire dei dati "default" con defaultDatabase(). Per la comunicazione con il database è utilizzata la classe AdminDAO.

\begin{figure}[H]
    \centering
    \includegraphics[scale=0.5]{resources/images/Snippets/snippet_resetdatabase.png}
    \captionsetup{labelformat=empty, labelsep=none}
    \caption{Snippet 3: metodo \code{resetDatabase()} nella classe \code{AdminExtraController}}
    \label{fig:snippet_resetdatabase}
\end{figure}

\subsection{ORM (Object-Relational Mapping)}
Nel package \code{main.java.ORM} (percorso \code{src/main/java/ORM}) sono implementate le interfacce per la comunicazione del sistema con un database. Per una gestione più coerente dei dati.\
Le classi "DAO" che appartengono a questo package si occupano di formulare e lanciare QUERY, eseguite tramite JDBC.

\subsubsection{ConnectionManager}
La classe si occupa di gestire la connessione al database per tutte le classi DAO tramite il metodo \hyperref[fig:snippet_connection_manager]{\code{getConnection()}}.
Essendo implementata seguendo il design pattern del \textit{Singleton} non è possibile che due DAO si colleghino contemporaneamente al database e quindi che si verifichino perdite di dati.\\
Inoltre questa classe contiene le informazioni esatte su l'URL, username e password per stabilire la connessione.

\begin{figure}[H]
    \centering
    \includegraphics[scale=0.5]{resources/images/Snippets/snippet_connection_manager.png}
    \captionsetup{labelformat=empty, labelsep=none}
    \caption{Snippet 4: metodi \code{getInstance} e \code{getConnection()} nella classe \code{ConnectionManager}}
    \label{fig:snippet_connection_manager}
\end{figure}

\subsubsection{ClienteDAO}
La classe è preposta all'accesso al sistema tramite la funzione \hyperref[fig:snippet_clienteDAO]{\code{accedi(email, password)}} e alla registrazione (\hyperref[fig:snippet_clienteDAO]{\code{registra(...)}}) di un nuovo cliente. In più sono presenti i metodi per la modifica delle informazioni (\code{modificaAttributo(id\_cliente, ...)}) e per restituire un cliente presente in database in base a criteri specificati (\code{get(criteri)}). \\
Per fare ciò (come visibile dallo \hyperref[fig:snippet_clienteDAO]{Snippet 5}) viene creata una stringa con i comandi in linguaggio SQL (\code{insertQuery}), creato il relativo \code{statement} e popolato con i valori delle variabili corrispondenti. Una volta eseguita la query e constatato che è andata a buon fine, è possibile ottenere eventuali dati e informazioni dal \code{ResultSet} della query.

\begin{figure}[H]
    \centering
    \includegraphics[scale=0.65]{resources/images/Snippets/snippet_clienteDAO.png}
    \captionsetup{labelformat=empty,labelsep=none}
    \caption{Snippet 5: metodi \code{registra()} e \code{accedi()} nella classe \code{ClienteDAO}}
    \label{fig:snippet_clienteDAO}
\end{figure}

\subsubsection{AdminDAO}
La classe svolge la funzione di accesso per l'admin come per il cliente ma ha anche i metodi   \\ \code{resetDatabase(sql), createDatabase(sql)} e \code{generateDefaultDatabase(sql)} che eseguono il reset.sql, schema.sql e defaul.sql tali da pulire e resettare il database.

\subsubsection{OperatoreDAO}
\code{OperatoreDAO} permette anch'esso l'accesso al sistema con funzione \code{accedi(email,password)} con valore di ritorno un booleano. Ha anche un metodo \code{getById(id)} per restituire l'operatore a partire dall'Id richiesto.

\subsubsection{SensoreDAO, AttuatoreDAO e OperazioneDAO}
Le classi \code{SensoreDAO} e \code{AttuatoreDAO} restituiscono i rispettivi oggetti tramtie \code{getById(id)} e hanno anche il metodo \hyperref[fig:snippet_sensoreDAOaggiorna]{\code{aggiorna(sensore)}} (e \code{aggiorna(attuatore)}) utilizzato per aggiornare i parametri delle tabelle. La classe \code{OperazioneDAO} permette di tenere un record delle azioni degli attuatori tramite la funzione \code{registra(attuatore, descrizione, data)}. \\
Nello \hyperref[fig:snippet_sensoreDAOaggiorna]{Snippet 6} è possibile vedere l'implementazione del metodo \code{aggiorna} di un \code{Sensore}: viene creato uno statement con la stringa della query, viene popolata con i dati e poi viene eseguito l'aggiornamento (\code{executeUpdate}) del database.

\begin{figure}[H]
    \centering
    \includegraphics[scale=0.5]{resources/images/Snippets/snippet_sensoreDAOaggiorna.png}
    \captionsetup{labelformat=empty,labelsep=none}
    \caption{Snippet 6: metodo \code{aggiorna()} nella classe \code{SensoreDAO}}
    \label{fig:snippet_sensoreDAOaggiorna}
\end{figure}

\subsubsection{SettoreDAO, PosizioneDAO}
Mentre \code{SettoreDAO} ha solo due metodi per la creazione del settore (\code{getById(id)} e\code{get(criteri)}), \code{PosizioneDAO} non sono ha questi metodi ma anche metodi per assegnare , aggiornare e verificare se ci sono posizioni non assegnate.

\subsubsection{OrdineDAO e PiantaDAO}
\code{OrdineDAO} e \code{PiantaDAO}, come le altre DAO, hanno i metodi per inserire, aggiornare e restituire gli oggetti associati, inoltre \code{PiantaDAO} ha metodi per eliminare elementi, usato quando l'ordine è espletato.\\
Le due DAO sono direttamente collegate in quanto nel momento di richiesta di dati di un \code{Ordine} a \code{OrdineDAO}, il suo metodo \hyperref[fig:snippet_OrdineDAOget]{\code{get}} invoca il metodo \hyperref[fig:snippet_piantaDAOget]{\code{get}} di un \code{PiantaDAO} in modo da inizializzare anche la lista di \code{Piante}, attributo della classe \code{Ordine}.

\begin{figure}[H]
    \centering
    \includegraphics[scale=0.5]{resources/images/Snippets/snippet_piantaDAOget.png}
    \captionsetup{labelformat=empty,labelsep=none}
    \caption{Snippet 7: metodo \code{get()} nella classe \code{PiantaDAO}}
    \label{fig:snippet_piantaDAOget}
\end{figure}

\begin{figure}[H]
    \centering
    \includegraphics[scale=0.5]{resources/images/Snippets/snippet_OrdineDAOget.png}
    \captionsetup{labelformat=empty,labelsep=none}
    \caption{Snippet 8: metodo \code{get()} nella classe \code{OrdineDAO}}
    \label{fig:snippet_OrdineDAOget}
\end{figure}

\newpage

\subsubsection{PosizionamentoDAO}
La classe \code{PosizionamentoDAO} permette di ottenere i Posizionamenti richiesti dal database, aggiungerne di nuovi o eliminarli in base all'id dell'ordine di cui fanno parte (\hyperref[fig:snippet_eliminaposizionamentibyordine]{\code{snippet\_eliminaposizionamentibyordine()}}).

\begin{figure}[H]
    \centering
    \includegraphics[scale=0.5]{resources/images/Snippets/snippet_eliminaposizionamentibyordine.png}
    \captionsetup{labelformat=empty,labelsep=none}
    \caption{Snippet 9: metodo \code{eliminaPosizionamentiByOrdine} della classe \code{PosizionamentoDAO}}
    \label{fig:snippet_eliminaposizionamentibyordine}
\end{figure}


\subsection{Database}
Per mantenere coerenza tra i dati del sistema è stato utilizzato un database utilizzando PostgreSQL, con il quale è possibile interfacciarsi tramite le classi DAO. Tale database segue lo schema delle operazioni CRUD (CREATE, READ, UPDATE, DELETE). \\
Per inizializzare il database sono stati creati 3 file \code{.sql}:
\begin{itemize}
    \item \code{reset.sql}: elimina tutte le tabelle del database (\code{DROP TABLE});
    \item \hyperref[fig:snippet_createtable]{\code{schema.sql}}: crea le tabelle del database definendo i vari attributi, chiavi e vincoli (\code{CREATE TABLE});
    \item \code{default.sql}: inserisce valori nelle tabelle in modo da avere uno stato iniziale già compatibile con il programma (\code{INSERT INTO}).
\end{itemize}
Questi 3 file sono stati eseguiti, grazie ai metodi presenti in AdminController, in fase di debug del programma e sono utilizzati anche prima di ogni test (\code{BeforeEach}) per impostare condizioni ben precise richieste dai test.

\begin{figure}[H]
    \centering
    \includegraphics[scale=0.5]{resources/images/Snippets/snippet_createtable.png}
    \captionsetup{labelformat=empty,labelsep=none}
    \caption{Snippet 10: creazione tabelle \code{Cliente}, \code{Ordine} e \code{Pianta} nel file \code{schema.sql}}
    \label{fig:snippet_createtable}
\end{figure}


\subsection{Interfaccia}
Ai fini di debug e test del programma è stata implementata un'interfaccia da terminale che permette di fare il login ed eseguire azioni/compiti sul sistema.\\
È possibile navigare tra i menu inserendo da tastiera il numero corrispondente all'opzione scelta.\\
Laddove è richiesto, l'utente deve fornire certe stringhe in input (ad esempio le credenziali di accesso per il login).\\
Il diagramma dei menu disponibili si trova nella sezione \ref{subsec:nav-diagram}.

\begin{figure}[H]
    \centering
    \includegraphics[scale=0.5]{resources/images/Snippets/snippet_handleclienteaction.png}
    \captionsetup{labelformat=empty,labelsep=none}
    \caption{Snippet 11: metodo \code{handleClienteAction} che mostra il menu principale del \code{Cliente}}
    \label{fig:snippet_handleclienteaction}
\end{figure}

\begin{figure}[H]
    \centering
    \includegraphics[scale=0.5]{resources/images/Snippets/snippet_handlesceltaordinedapiantare.png}
    \captionsetup{labelformat=empty,labelsep=none}
    \caption{Snippet 12: metodo \code{handleSceltaOrdineDaPiantare}}
    \label{fig:snippet_handlesceltaordinedapiantare}
\end{figure}


\section{Testing}
Per verificare la correttezza e funzionalità del programma sono stati realizzati dei test (vedi cartella \code{test} in \code{/src}). Questi riguardano quelle funzioni associate agli use cases obiettivo del progetto, quindi principalmente la BusinessLogic, che è la parte più "ad alto livello" del programma e che contiene le funzioni direttamente chiamate dall'interfaccia. \\

\noindent Le funzioni-test sono collegate gli Use Cases che rappresentano, alcuni in solo uno e più passaggi e altri totalmente.\\


\subsection{Business Logic Test}
Molti di questi test presentano più variazioni, in quanto verificano sia il caso in cui le condizioni per una certa funzione siano favorevoli, quindi è atteso un esito positivo ("Success"), sia il caso in cui vengano riscontrati problemi e quindi ci si aspetta un esito negativo ("Fail").\\

\noindent I test sono suddivisi in più classi (rispettando la suddivisione che hanno le relative funzioni): \\
\code{LoginClienteControllerTest}, \code{LoginPersonaleControllerTest}, 
\code{ClienteControllerTest},\\ 
\code{OperatoreControllerTest}.

\subsubsection{LoginClienteControllerTest}
Si effettuano i test delle funzioni utilizzate dal cliente per l'accesso e la registrazione al sistema quali:  
\begin{itemize}
    \item \code{setUp}: reimposta il database a uno stato iniziale default;
    \item \code{registrationTest\_Success}: esegue il test del metodo \code{registrati} di \code{loginClienteController}, utilizzata per la registrazione di un nuovo cliente (Use Case \hyperref[use_case_1]{\#1});
    \item \code{registrationTest\_Fail}: come \code{registrationTest\_Success}, ma l'\code{email} con il quale si tenta di registrarsi appartiene a un cliente già presente nel database, quindi il test di aspetta un fallimento nella procedura (Use Case  \hyperref[use_case_1]{\#1});
    \item \code{loginTest\_Success}: verifica la corretta funzionalità del metodo \code{accedi} che permette al cliente di accedere al sistema (Use Case \hyperref[use_case_2]{\#2});
    \item \code{loginTest\_Fail1} e \code{loginTest\_Fail2}: verifica l'effettivo fallimento del login nel caso in cui la \code{password} o l'\code{email} non siano corrette (Use Case \hyperref[use_case_2]{\#2});    
\end{itemize}

\begin{figure}[H]
    \centering
    \includegraphics[scale=0.5]{resources/images/Snippets/snippet_LoginClienteControllerTest.png}
    \captionsetup{labelformat=empty,labelsep=none}
    \caption{Snippet 13: test di \code{LoginClienteControllerTest}}
    \label{fig:snippet_LoginClienteControllerTest}
\end{figure}

\newpage

\subsubsection{LoginPersonaleControllerTest}
Analogo a \code{LoginClienteControllerTest}, effettua i test di accesso nel sistema, ma da parte del personale lavorativo (\code{Admin} e \code{Operatore}) e contiene i seguenti test: \code{registrationTest\_Fail}, \code{loginAdmin\_Fail1}, \code{loginAdmin\_Fail2}, \code{loginOperatore\_Success}, \code{loginOperatore\_Fail1} e \code{loginOperatore\_Fail2}
(vedi Use Case \hyperref[use_case_1]{\#1} e \hyperref[use_case_2]{\#2}).

\begin{figure}[H]
    \centering
    \includegraphics[scale=0.5]{resources/images/Snippets/snippet_LoginPersonaleControllerTest.png}
    \captionsetup{labelformat=empty,labelsep=none}
    \caption{Snippet 14: test di \code{LoginPersonaleControllerTest}}
    \label{fig:snippet_LoginPersonaleControllerTest}
\end{figure}

\subsubsection{ClienteControllerTest}
Verifica il corretto funzionamento dei metodi di \code{ClienteController}, quindi delle funzionalità a disposizione dei Clienti. Sono testate le funzioni di aggiornamento dei dati del profilo, richiesta nuovo ordine (Success e Fail), pagamento e ritiro di un ordine (Success e Fail);
\begin{itemize}
    \item \code{testAggiornaProfilo}: chiama il metodo \code{aggiornaProfilo} di \code{ClienteController}, e modifica il \code{nome} del cliente;
    \item \hypertarget{testRichiediNuovoOrdine\_Success}{\code{testRichiediNuovoOrdine\_Success}}: crea un nuovo ordine e invia la richiesta al sistema. Si aspetta che tale ordine sia accettato senza problemi (Use Case \hyperref[use_case_3]{\#3});
    \item \hypertarget{testRichiediNuovoOrdine\_Fail}{\code{testRichiediNuovoOrdine\_Fail}}: come il precedente, ma il numero di piante richiesto è superiore alla capacità dell'impianto, quindi la richiesta fallisce (Use Case \hyperref[use_case_3]{\#3});
    \item \hypertarget{testPagaERitiraOrdine}{\code{testPagaERitiraOrdine}}: verifica che il metodo \code{pagaERitiraOrdine} di \code{ClienteController} funzioni correttamente Use Case \hyperref[use_case_4]{\#4}).
\end{itemize}

\begin{figure}[H]
    \centering
    \includegraphics[scale=0.5]{resources/images/Snippets/snippet_ClienteControllerTest.png}
    \captionsetup{labelformat=empty,labelsep=none}
    \caption{Snippet 15: test di \code{ClienteControllerTest}}
    \label{fig:snippet_ClienteControllerTest}
\end{figure}

\subsubsection{OperatoreControllerTest}
\begin{itemize}
    \item \code{completaOrdineTest\_Success}: verifica che il completamento di un ordine venga eseguito correttamente (Use Case \hyperref[use_case_6]{\#6});
    \item \code{completaOrdineTest\_Fail}: prova ad eseguire il completamento di un ordine che non è ancora stato piantato, perciò si aspetta un fallimento (Use Case \hyperref[use_case_7]{\#7});
    \item \code{checkupPianteTest\_Success}: test del check-up delle piante eseguito da parte di un operatore. Alcune piante vengono (forzatamente) impostate come bisognose di cure e si verifica che queste vengano tutte rilevate durante il check-up (Use Case \hyperref[use_case_7]{\#7});
    \item \code{checkupPianteTest\_Fail}: nessuna pianta ha bisogno di cure (in base a questo setup), quindi il check-up non rileva piante bisognose (Use Case \hyperref[use_case_7]{\#7});
    \item \code{curaPiantaTest}: chiama il metodo \code{curaPianta} di \code{OperatoreController} e ne verifica il funzionamento (Use Case \hyperref[use_case_7]{\#7});
    \item \code{piantaOrdineTest}: testa la funzione utilizzata per piantare un ordine (Use Case \hyperref[use_case_5]{\#5}).
\end{itemize}

\begin{figure}[H]
    \centering
    \includegraphics[scale=0.65]{resources/images/Snippets/snippet_OperatoreControllerTest.png}
    \captionsetup{labelformat=empty,labelsep=none}
    \caption{Snippet 16: test di \code{OperatoreControllerTest}}
    \label{fig:snippet_OperatoreControllerTest}
\end{figure}

\subsection{Domain Model Test}
Per quanto riguarda il Domain Model è stato effettuato un solo test, in quanto sono presenti classi che fungono soltanto da entità, e generalmente sono passive, cioè prive di metodi e funzionalità.\\
I sensori però contengono un metodo \code{misura} che prende in ingresso l'orario e lo stato dell'attuatore associato a tale sensore e ritorna una misurazione del parametro fisico che gli compete. Questo simula l'influenza del tempo e degli attuatori sulla misurazione.\\
Nel caso in cui l'\code{Irrigatore} sia acceso (\hyperref[fig:snippet_IgrometroTerraTest]{\code{testMisura\_Irr\_ON}}), ci si aspetta un valore di umidità del terreno (misurato da \code{IgrometroTerra}) crescente nel tempo. Altrimenti (\hyperref[fig:snippet_IgrometroTerraTest]{\code{testMisura\_Irr\_OFF}}) il valore diminuisce (l'acqua viene assorbita dalla pianta). Il test verifica questo.

\begin{figure}[H]
    \centering
    \includegraphics[scale=0.5]{resources/images/Snippets/snippet_IgrometroTerraTest.png}
    \captionsetup{labelformat=empty,labelsep=none}
    \caption{Snippet 17: test di \code{IgrometroTerra}}
    \label{fig:snippet_IgrometroTerraTest}
\end{figure}

\subsection{ORM Test}
È stato inserito un test che riguarda la classe \code{OrdineDAO}, nel package ORM.\\
Questa classe, che si occupa della gestione degli Ordini nel database, invoca un'altra classe "DAO" al suo interno, in quanto ha la responsabilità di inizializzare l'oggetto della classe \code{Ordine}, che contiene l'attributo \code{piante}, una arraylist di \code{Piante}. \\
Quindi, come visibile nello \hyperref[fig:snippet_testInitPianteOrdineDAO]{Snippet 18}, questo test inserisce un ordine (\code{ordineRichiesto}) nel database tramite \code{OrdineDAO} e le relative piante con \code{PiantaDAO}; poi chiama il metodo \code{get} di \code{OrdineDAO} e verifica che i due ordini corrispondano.\\
In particolare verifica che le piante contenute nei due ordini siano dello stesso tipo e in quantità uguale (attraverso dei "contatori").\\
Se il test è superato, \code{OrdineDAO} ha inizializzato correttamente le piante.

\begin{figure}[H]
    \centering
    \includegraphics[scale=0.5]{resources/images/Snippets/snippet_testInitPianteOrdineDAO.png}
    \captionsetup{labelformat=empty,labelsep=none}
    \caption{Snippet 18: test di inizializzazione piante in \code{OrdineDAO}}
    \label{fig:snippet_testInitPianteOrdineDAO}
\end{figure}

\newpage\subsection*{Legenda dei Test}
Di seguito è presentata una legenda per i test che sono riconducibili agli Use Case analizzati.
\begin{table}[h!]
    \centering
    \begin{tabular}{>{\raggedright}p{0.3\textwidth} p{0.6\textwidth}}
        \textbf{Test ID} & \textbf{Descrizione Test} \\
        \hline
        Test \#1 & \hyperref[fig:snippet_LoginClienteControllerTest]{\code{registrationTest\_Success()}}
        \label{test_1}\\
        
        Test \#2 & \hyperref[fig:snippet_LoginClienteControllerTest]{\code{registrationTest\_Fail()}}
        \label{test_2}\\
        
        Test \#3 & \hyperref[fig:snippet_LoginClienteControllerTest]{\code{loginTest\_Success()}}
        \label{test_3}\\
        
        Test \#4 & \hyperref[fig:snippet_LoginClienteControllerTest]{\code{loginTest\_Fail1()}}
        \label{test_4}\\
        
        Test \#5 & \hyperref[fig:snippet_LoginClienteControllerTest]{\code{loginTest\_Fail2()}}
        \label{test_5}\\
        
        Test \#6 & \hyperref[fig:snippet_LoginPersonaleControllerTest]{\code{loginAdmin\_Success()}}
        \label{test_6}\\
        
        Test \#7 & \hyperref[fig:snippet_LoginPersonaleControllerTest]{\code{loginAdmin\_Fail1()}}
        \label{test_7}\\
        
        Test \#8 & \hyperref[fig:snippet_LoginPersonaleControllerTest]{\code{loginAdmin\_Fail2()}}
        \label{test_8}\\

        Test \#9 & \hyperref[fig:snippet_LoginPersonaleControllerTest]{\code{loginOperatore\_Success()}}
        \label{test_9}\\
        
        Test \#10 & \hyperref[fig:snippet_LoginPersonaleControllerTest]{\code{loginOperatore\_Fail1()}}
        \label{test_10}\\
        
        Test \#11 & \hyperref[fig:snippet_LoginPersonaleControllerTest]{\code{loginOperatore\_Fail2()}}
        \label{test_11}\\
        
        Test \#12 & \hyperref[fig:snippet_ClienteControllerTest]{\code{testRichiediNuovoOrdine\_Success()}}
        \label{test_12}\\
        
        Test \#13 & \hyperref[fig:snippet_ClienteControllerTest]{\code{testRichiediNuovoOrdine\_Fail()}}
        \label{test_13}\\

        Test \#14 & \hyperref[fig:snippet_ClienteControllerTest]{\code{testPagaERitiraOrdine()}}
        \label{test_14}\\

        Test \#15 & \hyperref[fig:snippet_OperatoreControllerTest]{\code{completaOrdineTest\_Success()}}
        \label{test_15}\\

        Test \#16 & \hyperref[fig:snippet_OperatoreControllerTest]{\code{completaOrdineTest\_Fail()}}
        \label{test_16}\\

        Test \#17 & \hyperref[fig:snippet_ClienteControllerTest]{\code{checkupPianteTest\_Success()}}
        \label{test_16}\\

        Test \#18 & \hyperref[fig:snippet_OperatoreControllerTest]{\code{checkupPianteTest\_Fail()}}
        \label{test_18}\\

        Test \#19 & \hyperref[fig:snippet_OperatoreControllerTest]{\code{curaPiantaTest()}}
        \label{test_19}\\

        Test \#20 & \hyperref[fig:snippet_OperatoreControllerTest]{\code{piantaOrdineTest()}}
        \label{test_20}\\
    \end{tabular}
    \caption{Legenda dei Test}
\end{table}

\end{document}